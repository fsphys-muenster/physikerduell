\documentclass{scrartcl}
% -- Preamble
\renewcommand{\familydefault}{\rmdefault}		% change default font family
\usepackage[T1]{fontenc}						% change font encoding to 8 bit (256 glyphs)
\usepackage[utf8]{inputenc}					% set encoding to UTF8
\usepackage{lmodern}
\usepackage[ngerman]{babel}

\usepackage{graphicx}		% add graphic images
\usepackage{color}			% add text colours

\usepackage{scrpage2}		% support for advanced editing of header and footer

% -- create header
\pagestyle{scrheadings}
\ohead[]{Fachschaft Physik der WWU\\
Lutz Althüser, Simon May}						% right column
\ihead[]{Physikerduell Regelwerk\\ 2014}	% left column
\setheadsepline{1pt} 
\setlength{\headheight}{30pt}
% -- header end

% -- main document
\begin{document}
\section*{Physikerduell Regelwerk}

\subsection*{Aufbau}
Für das Physikerduell wird ein Moderator benötigt, der Gäste und Publikum durch die Show führt und die Entscheidungsmacht über die Gültigkeit von Antworten besitzt. Für den Eins-gegen-Eins-Part wird eine Klingel, ein Button oder ähnliches benötigt. Das Spiel wird über ein Computerprogramm visualisiert und von einer darin erfahrenen Person administriert.

\subsection*{Die Teams}
Es werden zwei oder mehr Teams von fünf Personen gebildet. Jedes dieser Teams besteht aus Mitgliedern einer Arbeitsgruppe und im optimalen Fall dem Professor derselben.

Gibt es mehr als zwei Teams, werden zunächst die Gruppensieger und anschließend in einem "`Finale"' der Gesamtsieger ermittelt.

\subsection*{Spielablauf}
Es gibt 4 normale Runden und die Finalrunde. Für jeder dieser Runden gelten spezielle Regeln bezüglich der Punktezahlen und Antwortmöglichkeiten.
\begin{itemize}
\item Runden 1 und 2: Es gilt, die sechs häufigsten Antworten zu finden. Gespielt wird um die einfache Punktzahl.
\item Runde 3: Es gilt, die fünf häufigsten Antworten zu finden. Gespielt wird um die doppelte Punktzahl.
\item Runde 4: Es gilt, die vier häufigsten Antworten zu finden. Gespielt wird um die doppelte Punktzahl.
\item Finale: Es gilt, die drei häufigsten Antworten zu finden. Gespielt wird um die dreifache Punktzahl.
\end{itemize}
Zu Beginn jeder Runde kommt ein anderes Teammitglied nach vorne an den "`Buzzertisch"'. Dort kämpfen die Vertreter der beiden Teams darum, bei einer Frage die erste Antwort geben zu dürfen. Derjenige, der zuerst den Buzzer berührt, darf zuerst antworten. Hat dieser die Top-Antwort (höchste Punktzahl) gegeben, geht die Runde an seine Arbeitsgruppe. Wenn er nicht die Top-Antwort gegeben hat, darf der andere Gegenspieler eine Antwort geben. Gibt diese Antwort mehr Punkte, so geht die Runde an diese Gruppe.

Von jetzt an wird die zuvor ausgewählte Arbeitsgruppe weiter durchgefragt. Sobald zwei falsche Antworten gegeben wurden, darf sich die andere Arbeitsgruppe beraten. Ist schließlich auch die 3. falsche Antwort gegeben, so ist es der anderen Arbeitsgruppe nun möglich, die Punkte zu "`klauen"'. Dazu sagt jeder aus der Gruppe eine Antwort und der Gruppenleiter entscheidet über die Antwort, die eingeloggt werden soll. Sollte diese Antwort auf der Liste sein, so konnten alle Punkte geklaut werden, d.h. die während der Runde erspielten Punkte gehen an das andere Team. Andernfalls behält das vorher spielende Team die erspielten Punkte. Wird die benötigte Zahl an Antworten (s.o.) gefunden, erhält das spielende Team einfach alle erspielten Punkte.

Damit sind die Fragen an das erste Team beendet. Die Rollen der beiden Teams tauschen nun: Das zweite Team spielt und nach 3 falschen Antworten kann das erste Team versuchen, Punkte zu stehlen (oder es werden alle Antworten gefunden). Ist auch das zweite Team fertig, wird wieder per Buzzer über das startende Team entschieden und es folgt die nächste Runde nach dem gleichen Ablauf.

\subsection*{Gewinn}
Das Gewinnerteam bekommt einen Pokal zur Würdigung der Leistung.

\end{document}
